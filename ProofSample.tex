\documentclass[12pt]{amsart}
\usepackage[onehalfspacing]{setspace}
\usepackage{amsfonts}
\usepackage{amssymb}

\usepackage[english]{babel}
\usepackage[utf8]{inputenc}
\usepackage{amsmath}
\usepackage{mathrsfs}
\usepackage{verbatim}
\usepackage{amsthm}


\newtheorem{theorem}{Theorem}
\newtheorem{definition}{Definition}
\newtheorem{lemma}{Lemma}[section]

\title{Proof Sample}

\author{M.~J.~Klein}

\date{\today}

\begin{comment}

The MIT License (MIT)

Copyright (c) 2015 michaeljklein

Permission is hereby granted, free of charge, to any person obtaining a copy
of this software and associated documentation files (the "Software"), to deal
in the Software without restriction, including without limitation the rights
to use, copy, modify, merge, publish, distribute, sublicense, and/or sell
copies of the Software, and to permit persons to whom the Software is
furnished to do so, subject to the following conditions:

The above copyright notice and this permission notice shall be included in all
copies or substantial portions of the Software.

THE SOFTWARE IS PROVIDED "AS IS", WITHOUT WARRANTY OF ANY KIND, EXPRESS OR
IMPLIED, INCLUDING BUT NOT LIMITED TO THE WARRANTIES OF MERCHANTABILITY,
FITNESS FOR A PARTICULAR PURPOSE AND NONINFRINGEMENT. IN NO EVENT SHALL THE
AUTHORS OR COPYRIGHT HOLDERS BE LIABLE FOR ANY CLAIM, DAMAGES OR OTHER
LIABILITY, WHETHER IN AN ACTION OF CONTRACT, TORT OR OTHERWISE, ARISING FROM,
OUT OF OR IN CONNECTION WITH THE SOFTWARE OR THE USE OR OTHER DEALINGS IN THE
SOFTWARE.

\end{comment}

\begin{document}

\maketitle
\pagestyle{plain}

\section{Introduction}
	This problem came up in the process of searching for lower bounds on the number of cycles in a graph with certain properties.
	
	I chose this proof as my sample because it's my favorite short proof. The final bit of reasoning is analogous to reducing the problem to a halting problem and solving that halting problem. 
\section{Proof}

This is a proof of the lower bound of the pseudo-boolean function 
	$$f(A) := \sum_{i=1}^{m-1} \left( \sum_{j=1}^{i} a_j \right)^2$$ 
	given that $A \in \mathbb{B}^{m-1}$, $\varepsilon = \sum_{i=1}^{m-1} a_i = \text{constant}$, and $m \geq 2$. 
	In other words, this is an optimization of the function $f : \mathbb{B}^{m-1} \to \mathbb{Z}$, where the binary input vector has a given Hamming Weight.

\begin{lemma}
\label{lem10}
	$$\sum_{i = 1}^{m-1} \sum_{j=1}^{i} a_j = m \sum_{i = 1}^{m-1} a_i - \sum_{i = 1}^{m-1} i a_i$$
\end{lemma}
\begin{proof}
	$$\sum_{i = 1}^{m-1} \sum_{j=1}^{i} a_j = (a_1) + (a_1 + a_2) + (a_1 + a_2 + a_3) + \dots$$
	We have $(m-i)$ copies of $a_i$ so \\
	$$\sum_{i = 1}^{m-1} \sum_{j=1}^{i} a_j = \sum_{i = 1}^{m-1} (m - i) a_i = m \sum_{i = 1}^{m-1} a_i - \sum_{i = 1}^{m-1} i a_i$$
\end{proof}


\begin{lemma}
\label{lem11}
	$$\sum_{i = 1}^{m-1} \sum_{j=1}^{i} a_j i
	= \frac{m^2}{2} \sum_{i=1}^{m-1} a_i
	- \frac{m}{2} \sum_{i=1}^{m-1} a_i
	+ \frac{1}{2} \sum_{i=1}^{m-1} i a_i
	- \frac{1}{2} \sum_{i=1}^{m-1} i^2 a_i$$
\end{lemma}
\begin{proof}
	\begin{align*}
		\sum_{i = 1}^{m-1} \sum_{j=1}^{i} a_j i &= 1(a_1) + 2(a_1 + a_2) + 3(a_1 + a_2 + a_3) + \dots \\
		&= a_1 \sum_{j=1}^{m-1} j + a_2 \sum_{j=2}^{m-1} j + a_3 \sum_{j=3}^{m-1} j + \dots
	\end{align*}
	But $\sum_{j=i}^{m-1} j = (m - i) (m + i - 1)/2$ so then we have:
	\begin{align*}
		\sum_{i = 1}^{m-1} \sum_{j=1}^{i} a_j i &= \sum_{i=1}^{m-1} a_i (m - i) (m + i - 1)/2 \\
									  &= \frac{1}{2} \sum_{i=1}^{m-1} a_i (m^2 - m + i - i^2) \\
									  &= \frac{m^2}{2} \sum_{i=1}^{m-1} a_i
	- \frac{m}{2} \sum_{i=1}^{m-1} a_i
	+ \frac{1}{2} \sum_{i=1}^{m-1} i a_i
	- \frac{1}{2} \sum_{i=1}^{m-1} i^2 a_i
	\end{align*}
\end{proof}

\begin{lemma}
\label{lem12}
	$$\sum_{i=1}^{m} \left( \sum_{j=1}^{i} a_{m+1-j} \right)^2 = \sum_{i=1}^{m} i a_i  + \sum_{i=2}^{m} \left( \sum_{j=1}^{i-1} 2 j a_j a_i \right)$$
\end{lemma}
\begin{proof}
	Note that 
		%
	$$\left( \sum_{j=1}^{i} a_{m+1-j} \right)^2 = \sum_{j=1}^{i} a_{m+1-j}^2 + 2 \sum_{j<k}^{i} a_{m+1-j} a_{m+1-k}$$
		%
	But $a_{m+1-j}^2 = a_{m+1-j}$, because $a_{m+1-j} \in \{ 0, 1 \}$ so:
		%
	$$\sum_{j=1}^{i} a_{m+1-j}^2 = \sum_{j=1}^{i} a_{m+1-j}$$
		%
	Thus 
		%
	$$\sum_{i=1}^{m} \left( \sum_{j=1}^{i} a_{m+1-j} \right)^2 = 
		   \sum_{i=1}^{m} \left( \sum_{j=1}^{i} a_{m+1-j}                     \right)
		+ \sum_{i=1}^{m} 2 \left( \sum_{j<k}^{i} a_{m+1-j} a_{m+1-k} \right)
	$$
		%
	But note that, because there are $i$ copies of $a_{m+1-(m+1-i)}$,
		%
	$$\sum_{i=1}^{m} \left( \sum_{j=1}^{i} a_{m+1-j} \right) = \sum_{i=1}^{m} i a_i$$
		%
	Likewise, there are $i$ copies of each distinct pair of $a$'s, so
		%
	$$\sum_{i=1}^{m} 2 \left( \sum_{j<k}^{i} a_{m+1-j} a_{m+1-k} \right) = \sum_{i=1}^{m} \left( \sum_{j=2}^{i-1} 2 j a_j a_i \right)$$
		%
\end{proof} 


\begin{theorem}
\label{lem13}
	$$\frac{1}{6} \varepsilon (\varepsilon + 1) (2 \varepsilon+ 1) \leq \sum_{i=1}^{m-1} \left( \sum_{j=1}^{i} a_j \right)^2 = f(A)$$
\end{theorem}
\begin{proof}
	We begin by reversing the indices of the $a_j$'s and applying Lemma \ref{lem12} to get
		%
	$$\sum_{i=1}^{m-1} \left( \sum_{j=1}^{i} a_j \right)^2 = \sum_{i=1}^{m} i a_i  + \sum_{i=2}^{m} \left( \sum_{j=1}^{i-1} 2 j a_j a_i \right)$$
		%
	Recall that $\varepsilon = \sum_{i=1}^{m-1} a_i$.
	We first consider the trivial cases of $\varepsilon = 0, 1$.
	
		If $\varepsilon = 0$, then all the $a_i = 0$ and so the total is $0$ and the theorem holds.

		If $\varepsilon = 1$, then exactly one $a_i \neq 0$, lets call it $a_p$ for positive. 
		Then the total is equal to $k a_k$, because no pair of $a_i$'s can be non-zero. 
		Taking the minimum $k = 1$, we get the total is $1$ and the theorem holds.

	Note that if $a_z$ is positive, it adds exactly $z + 2 z \sum_{j=z+1}^{m} a_j$ to the total, independent of the rest.
		
	Suppose that there exists some $y$ such that $a_y = 1$ and for all $a_i = 0$, $i < y$. Let $y$ be the minimum such index. 
		The contribution of $a_y$ is $y + 2 y \sum_{j=y+1}^{m} a_j$. Similarly the contribution of $a_{y-1}$, if we swap the values of $a_y$ and $a_{y-1}$, is $(y-1) + 2 (y-1) \sum_{j=y}^{m} a_j$. However, if we only swap these two values, the values, and thus the contributions of those values, with index greater than $y$ are left unchanged. 
		Therefore, the contribution of $a_{y-1}$, if the values are swapped, is \\ $(y-1) + 2 (y-1) \sum_{j=y+1}^{m} a_j$.
		As this operation of swapping the first positive value above all the zero-values will always decrease the total, one may repeat this operation as long as there are positive values with higher index than some zero value.
		Therefore, the minimum total is exactly the assignment where there is no non-positive assignment below a positive one, that is precisely the assignment with all positive assignments at the lowest possible indices.

	Thus our minimum total, given that $\varepsilon < i$ implies that $a_i = 0$, is
		%
	$$\sum_{i=1}^{\varepsilon} i + \sum_{i=2}^{\varepsilon} \left( \sum_{j=1}^{i-1} 2 j a_j a_i \right) = \frac{1}{6} \varepsilon (\varepsilon + 1) (2 \varepsilon+ 1)$$
		%
\end{proof}


\end{document}